% Options for packages loaded elsewhere
\PassOptionsToPackage{unicode}{hyperref}
\PassOptionsToPackage{hyphens}{url}
\PassOptionsToPackage{dvipsnames,svgnames,x11names}{xcolor}
%
\documentclass[
  12pt,
]{article}

\usepackage{amsmath,amssymb}
\usepackage{iftex}
\ifPDFTeX
  \usepackage[T1]{fontenc}
  \usepackage[utf8]{inputenc}
  \usepackage{textcomp} % provide euro and other symbols
\else % if luatex or xetex
  \usepackage{unicode-math}
  \defaultfontfeatures{Scale=MatchLowercase}
  \defaultfontfeatures[\rmfamily]{Ligatures=TeX,Scale=1}
\fi
\usepackage{lmodern}
\ifPDFTeX\else  
    % xetex/luatex font selection
\fi
% Use upquote if available, for straight quotes in verbatim environments
\IfFileExists{upquote.sty}{\usepackage{upquote}}{}
\IfFileExists{microtype.sty}{% use microtype if available
  \usepackage[]{microtype}
  \UseMicrotypeSet[protrusion]{basicmath} % disable protrusion for tt fonts
}{}
\makeatletter
\@ifundefined{KOMAClassName}{% if non-KOMA class
  \IfFileExists{parskip.sty}{%
    \usepackage{parskip}
  }{% else
    \setlength{\parindent}{0pt}
    \setlength{\parskip}{6pt plus 2pt minus 1pt}}
}{% if KOMA class
  \KOMAoptions{parskip=half}}
\makeatother
\usepackage{xcolor}
\usepackage[left=1in,right=1in,top=1in,bottom=1in]{geometry}
\setlength{\emergencystretch}{3em} % prevent overfull lines
\setcounter{secnumdepth}{5}
% Make \paragraph and \subparagraph free-standing
\makeatletter
\ifx\paragraph\undefined\else
  \let\oldparagraph\paragraph
  \renewcommand{\paragraph}{
    \@ifstar
      \xxxParagraphStar
      \xxxParagraphNoStar
  }
  \newcommand{\xxxParagraphStar}[1]{\oldparagraph*{#1}\mbox{}}
  \newcommand{\xxxParagraphNoStar}[1]{\oldparagraph{#1}\mbox{}}
\fi
\ifx\subparagraph\undefined\else
  \let\oldsubparagraph\subparagraph
  \renewcommand{\subparagraph}{
    \@ifstar
      \xxxSubParagraphStar
      \xxxSubParagraphNoStar
  }
  \newcommand{\xxxSubParagraphStar}[1]{\oldsubparagraph*{#1}\mbox{}}
  \newcommand{\xxxSubParagraphNoStar}[1]{\oldsubparagraph{#1}\mbox{}}
\fi
\makeatother

\usepackage{color}
\usepackage{fancyvrb}
\newcommand{\VerbBar}{|}
\newcommand{\VERB}{\Verb[commandchars=\\\{\}]}
\DefineVerbatimEnvironment{Highlighting}{Verbatim}{commandchars=\\\{\}}
% Add ',fontsize=\small' for more characters per line
\usepackage{framed}
\definecolor{shadecolor}{RGB}{241,243,245}
\newenvironment{Shaded}{\begin{snugshade}}{\end{snugshade}}
\newcommand{\AlertTok}[1]{\textcolor[rgb]{0.68,0.00,0.00}{#1}}
\newcommand{\AnnotationTok}[1]{\textcolor[rgb]{0.37,0.37,0.37}{#1}}
\newcommand{\AttributeTok}[1]{\textcolor[rgb]{0.40,0.45,0.13}{#1}}
\newcommand{\BaseNTok}[1]{\textcolor[rgb]{0.68,0.00,0.00}{#1}}
\newcommand{\BuiltInTok}[1]{\textcolor[rgb]{0.00,0.23,0.31}{#1}}
\newcommand{\CharTok}[1]{\textcolor[rgb]{0.13,0.47,0.30}{#1}}
\newcommand{\CommentTok}[1]{\textcolor[rgb]{0.37,0.37,0.37}{#1}}
\newcommand{\CommentVarTok}[1]{\textcolor[rgb]{0.37,0.37,0.37}{\textit{#1}}}
\newcommand{\ConstantTok}[1]{\textcolor[rgb]{0.56,0.35,0.01}{#1}}
\newcommand{\ControlFlowTok}[1]{\textcolor[rgb]{0.00,0.23,0.31}{\textbf{#1}}}
\newcommand{\DataTypeTok}[1]{\textcolor[rgb]{0.68,0.00,0.00}{#1}}
\newcommand{\DecValTok}[1]{\textcolor[rgb]{0.68,0.00,0.00}{#1}}
\newcommand{\DocumentationTok}[1]{\textcolor[rgb]{0.37,0.37,0.37}{\textit{#1}}}
\newcommand{\ErrorTok}[1]{\textcolor[rgb]{0.68,0.00,0.00}{#1}}
\newcommand{\ExtensionTok}[1]{\textcolor[rgb]{0.00,0.23,0.31}{#1}}
\newcommand{\FloatTok}[1]{\textcolor[rgb]{0.68,0.00,0.00}{#1}}
\newcommand{\FunctionTok}[1]{\textcolor[rgb]{0.28,0.35,0.67}{#1}}
\newcommand{\ImportTok}[1]{\textcolor[rgb]{0.00,0.46,0.62}{#1}}
\newcommand{\InformationTok}[1]{\textcolor[rgb]{0.37,0.37,0.37}{#1}}
\newcommand{\KeywordTok}[1]{\textcolor[rgb]{0.00,0.23,0.31}{\textbf{#1}}}
\newcommand{\NormalTok}[1]{\textcolor[rgb]{0.00,0.23,0.31}{#1}}
\newcommand{\OperatorTok}[1]{\textcolor[rgb]{0.37,0.37,0.37}{#1}}
\newcommand{\OtherTok}[1]{\textcolor[rgb]{0.00,0.23,0.31}{#1}}
\newcommand{\PreprocessorTok}[1]{\textcolor[rgb]{0.68,0.00,0.00}{#1}}
\newcommand{\RegionMarkerTok}[1]{\textcolor[rgb]{0.00,0.23,0.31}{#1}}
\newcommand{\SpecialCharTok}[1]{\textcolor[rgb]{0.37,0.37,0.37}{#1}}
\newcommand{\SpecialStringTok}[1]{\textcolor[rgb]{0.13,0.47,0.30}{#1}}
\newcommand{\StringTok}[1]{\textcolor[rgb]{0.13,0.47,0.30}{#1}}
\newcommand{\VariableTok}[1]{\textcolor[rgb]{0.07,0.07,0.07}{#1}}
\newcommand{\VerbatimStringTok}[1]{\textcolor[rgb]{0.13,0.47,0.30}{#1}}
\newcommand{\WarningTok}[1]{\textcolor[rgb]{0.37,0.37,0.37}{\textit{#1}}}

\providecommand{\tightlist}{%
  \setlength{\itemsep}{0pt}\setlength{\parskip}{0pt}}\usepackage{longtable,booktabs,array}
\usepackage{calc} % for calculating minipage widths
% Correct order of tables after \paragraph or \subparagraph
\usepackage{etoolbox}
\makeatletter
\patchcmd\longtable{\par}{\if@noskipsec\mbox{}\fi\par}{}{}
\makeatother
% Allow footnotes in longtable head/foot
\IfFileExists{footnotehyper.sty}{\usepackage{footnotehyper}}{\usepackage{footnote}}
\makesavenoteenv{longtable}
\usepackage{graphicx}
\makeatletter
\newsavebox\pandoc@box
\newcommand*\pandocbounded[1]{% scales image to fit in text height/width
  \sbox\pandoc@box{#1}%
  \Gscale@div\@tempa{\textheight}{\dimexpr\ht\pandoc@box+\dp\pandoc@box\relax}%
  \Gscale@div\@tempb{\linewidth}{\wd\pandoc@box}%
  \ifdim\@tempb\p@<\@tempa\p@\let\@tempa\@tempb\fi% select the smaller of both
  \ifdim\@tempa\p@<\p@\scalebox{\@tempa}{\usebox\pandoc@box}%
  \else\usebox{\pandoc@box}%
  \fi%
}
% Set default figure placement to htbp
\def\fps@figure{htbp}
\makeatother

\usepackage{setspace}
\doublespacing
\usepackage{float}
\floatplacement{figure}{t}
\floatplacement{table}{t}
\usepackage{flafter}
\usepackage[T1]{fontenc}
\usepackage[utf8]{inputenc}
\usepackage{ragged2e}
\usepackage{booktabs}
\usepackage{amsmath}
\usepackage{url}
\makeatletter
\@ifpackageloaded{caption}{}{\usepackage{caption}}
\AtBeginDocument{%
\ifdefined\contentsname
  \renewcommand*\contentsname{Table of contents}
\else
  \newcommand\contentsname{Table of contents}
\fi
\ifdefined\listfigurename
  \renewcommand*\listfigurename{List of Figures}
\else
  \newcommand\listfigurename{List of Figures}
\fi
\ifdefined\listtablename
  \renewcommand*\listtablename{List of Tables}
\else
  \newcommand\listtablename{List of Tables}
\fi
\ifdefined\figurename
  \renewcommand*\figurename{Figure}
\else
  \newcommand\figurename{Figure}
\fi
\ifdefined\tablename
  \renewcommand*\tablename{Table}
\else
  \newcommand\tablename{Table}
\fi
}
\@ifpackageloaded{float}{}{\usepackage{float}}
\floatstyle{ruled}
\@ifundefined{c@chapter}{\newfloat{codelisting}{h}{lop}}{\newfloat{codelisting}{h}{lop}[chapter]}
\floatname{codelisting}{Listing}
\newcommand*\listoflistings{\listof{codelisting}{List of Listings}}
\makeatother
\makeatletter
\makeatother
\makeatletter
\@ifpackageloaded{caption}{}{\usepackage{caption}}
\@ifpackageloaded{subcaption}{}{\usepackage{subcaption}}
\makeatother

\usepackage{bookmark}

\IfFileExists{xurl.sty}{\usepackage{xurl}}{} % add URL line breaks if available
\urlstyle{same} % disable monospaced font for URLs
\hypersetup{
  pdftitle={ProbSet 7, February 27},
  pdfauthor={Gideon Gordon; gideongordon2029@u.northwestern.edu},
  colorlinks=true,
  linkcolor={blue},
  filecolor={Maroon},
  citecolor={Blue},
  urlcolor={blue},
  pdfcreator={LaTeX via pandoc}}


\title{ProbSet 7, February 27}
\author{Gideon
Gordon \and \href{mailto:gideongordon2029@u.northwestern.edu}{\nolinkurl{gideongordon2029@u.northwestern.edu}}}
\date{\textbf{Due:} February 27, 2026}

\begin{document}
\maketitle


\textbf{Submission}:
\url{https://canvas.northwestern.edu/courses/245562/assignments/1687752}

\begin{Shaded}
\begin{Highlighting}[]
\FunctionTok{library}\NormalTok{(dplyr)}
\end{Highlighting}
\end{Shaded}

\begin{verbatim}

Attaching package: 'dplyr'
\end{verbatim}

\begin{verbatim}
The following objects are masked from 'package:stats':

    filter, lag
\end{verbatim}

\begin{verbatim}
The following objects are masked from 'package:base':

    intersect, setdiff, setequal, union
\end{verbatim}

\begin{Shaded}
\begin{Highlighting}[]
\FunctionTok{library}\NormalTok{(ggplot2)}
\end{Highlighting}
\end{Shaded}

\begin{verbatim}
Warning: package 'ggplot2' was built under R version 4.5.2
\end{verbatim}

\begin{center}\rule{0.5\linewidth}{0.5pt}\end{center}

\section{Problem 1}\label{problem-1}

\begin{enumerate}
\def\labelenumi{\arabic{enumi}.}
\item
  Define statistical power in your own words. Statistical power is the
  likelihood that, given a certain experimental setup, a certain
  acceptable alpha-level, and a certain actual effect size, the
  researcher will \emph{correctly} reject the null hypothesis. That is,
  the experimental setup can correctly determine that the effect is
  statistically different from zero.
\item
  Explain the relationship between Type I error (\(\alpha\)), Type II
  error (\(\beta\)), and power. A Type I error is where we incorrectly
  believe that a relationship is real when it is not (we incorrectly
  reject the null hypothesis). A Type II error is where we incorrectly
  believe that a relationship is not real when it totally is (we
  incorrectly fail to reject the null hypothesis).
\end{enumerate}

\begin{center}\rule{0.5\linewidth}{0.5pt}\end{center}

\section{Problem 2}\label{problem-2}

\subsection{2a.}\label{a.}

Simulate power for different scenarios:

\begin{Shaded}
\begin{Highlighting}[]
\CommentTok{\# Complete this code to simulate power for different sample sizes}
  \CommentTok{\# 1. Simulate n\_sim datasets with given parameters}
  \CommentTok{\# 2. For each dataset, run linear regression}
  \CommentTok{\# 3. Calculate proportion of simulations where p \textless{} alpha}
  \CommentTok{\# 4. Return power estimate}
\FunctionTok{set.seed}\NormalTok{(}\DecValTok{144}\NormalTok{)}

\NormalTok{simulate\_power }\OtherTok{\textless{}{-}} \ControlFlowTok{function}\NormalTok{(true\_effect, sample\_size, }\AttributeTok{sigma =} \DecValTok{1}\NormalTok{, }\AttributeTok{alpha =} \FloatTok{0.05}\NormalTok{, }\AttributeTok{n\_sim =} \DecValTok{1000}\NormalTok{) \{}
\NormalTok{  significant\_count }\OtherTok{=} \DecValTok{0}
  
  \ControlFlowTok{for}\NormalTok{ (i }\ControlFlowTok{in} \DecValTok{1}\SpecialCharTok{:}\NormalTok{n\_sim) \{}
\NormalTok{  x }\OtherTok{=} \FunctionTok{c}\NormalTok{(}\FunctionTok{rnorm}\NormalTok{(}\AttributeTok{n =}\NormalTok{ sample\_size))}
\NormalTok{  y }\OtherTok{=} \FunctionTok{c}\NormalTok{(true\_effect}\SpecialCharTok{*}\NormalTok{x }\SpecialCharTok{+} \FunctionTok{rnorm}\NormalTok{(}\AttributeTok{n =}\NormalTok{ sample\_size, }\AttributeTok{sd =}\NormalTok{ sigma))}
\NormalTok{  simulated\_data }\OtherTok{=} \FunctionTok{data.frame}\NormalTok{(}\AttributeTok{x =}\NormalTok{ x, }\AttributeTok{y =}\NormalTok{ y)}
\NormalTok{  simulated\_model }\OtherTok{=} \FunctionTok{lm}\NormalTok{(y }\SpecialCharTok{\textasciitilde{}}\NormalTok{ x, }\AttributeTok{data =}\NormalTok{ simulated\_data)}
\NormalTok{  p\_value\_i }\OtherTok{=} \FunctionTok{summary}\NormalTok{(simulated\_model)}\SpecialCharTok{$}\NormalTok{coefficients[}\DecValTok{2}\NormalTok{, }\DecValTok{4}\NormalTok{]}
  \ControlFlowTok{if}\NormalTok{ (p\_value\_i }\SpecialCharTok{\textless{}}\NormalTok{ alpha) \{significant\_count }\OtherTok{=}\NormalTok{ significant\_count }\SpecialCharTok{+} \DecValTok{1}\NormalTok{\}}
\NormalTok{  \}}
\NormalTok{  significant\_count }\SpecialCharTok{/}\NormalTok{ n\_sim}
\NormalTok{\}}

\FunctionTok{simulate\_power}\NormalTok{(}\FloatTok{0.2}\NormalTok{, }\DecValTok{100}\NormalTok{, }\AttributeTok{sigma =} \DecValTok{1}\NormalTok{, }\AttributeTok{alpha =} \FloatTok{0.05}\NormalTok{, }\AttributeTok{n\_sim =} \DecValTok{1000}\NormalTok{)}
\end{Highlighting}
\end{Shaded}

\begin{verbatim}
[1] 0.51
\end{verbatim}

\begin{Shaded}
\begin{Highlighting}[]
\CommentTok{\# Test for different sample sizes}
\NormalTok{sample\_sizes }\OtherTok{\textless{}{-}} \FunctionTok{c}\NormalTok{(}\DecValTok{50}\NormalTok{, }\DecValTok{100}\NormalTok{, }\DecValTok{200}\NormalTok{, }\DecValTok{400}\NormalTok{, }\DecValTok{800}\NormalTok{)}
\NormalTok{true\_effects }\OtherTok{\textless{}{-}} \FunctionTok{c}\NormalTok{(}\FloatTok{0.2}\NormalTok{, }\FloatTok{0.4}\NormalTok{)}
\NormalTok{sigmas }\OtherTok{=} \FunctionTok{c}\NormalTok{(}\DecValTok{1}\NormalTok{, }\DecValTok{2}\NormalTok{)}
\NormalTok{n\_sim }\OtherTok{=} \DecValTok{1000}
\end{Highlighting}
\end{Shaded}

\begin{Shaded}
\begin{Highlighting}[]
\CommentTok{\# Create a data frame with power estimates for each sample size}
\NormalTok{simulated\_power\_sample\_size }\OtherTok{=} \FunctionTok{data.frame}\NormalTok{(sample\_sizes) }

\NormalTok{count }\OtherTok{=} \DecValTok{0}
\NormalTok{powers\_0}\FloatTok{.2} \OtherTok{=} \FunctionTok{c}\NormalTok{()}
\ControlFlowTok{for}\NormalTok{ (i }\ControlFlowTok{in}\NormalTok{ sample\_sizes)\{}
\NormalTok{  count }\OtherTok{=}\NormalTok{ count }\SpecialCharTok{+} \DecValTok{1}
\NormalTok{  power }\OtherTok{=} \FunctionTok{simulate\_power}\NormalTok{(}\AttributeTok{true\_effect =} \FloatTok{0.2}\NormalTok{, }\AttributeTok{sample\_size =}\NormalTok{ sample\_sizes[count], }\AttributeTok{sigma =} \DecValTok{1}\NormalTok{, }\AttributeTok{alpha =} \FloatTok{0.05}\NormalTok{, }\AttributeTok{n\_sim =} \DecValTok{1000}\NormalTok{)}
\NormalTok{  powers\_0}\FloatTok{.2} \OtherTok{=} \FunctionTok{c}\NormalTok{(powers\_0}\FloatTok{.2}\NormalTok{, power)}
\NormalTok{\}}
\NormalTok{simulated\_power\_sample\_size\_0}\FloatTok{.2} \OtherTok{=} \FunctionTok{data.frame}\NormalTok{(sample\_sizes, powers\_0}\FloatTok{.2}\NormalTok{)}

\NormalTok{count }\OtherTok{=} \DecValTok{0}
\NormalTok{powers\_0}\FloatTok{.4} \OtherTok{=} \FunctionTok{c}\NormalTok{()}
\ControlFlowTok{for}\NormalTok{ (i }\ControlFlowTok{in}\NormalTok{ sample\_sizes)\{}
\NormalTok{  count }\OtherTok{=}\NormalTok{ count }\SpecialCharTok{+} \DecValTok{1}
\NormalTok{  power }\OtherTok{=} \FunctionTok{simulate\_power}\NormalTok{(}\AttributeTok{true\_effect =} \FloatTok{0.4}\NormalTok{, }\AttributeTok{sample\_size =}\NormalTok{ sample\_sizes[count], }\AttributeTok{sigma =} \DecValTok{1}\NormalTok{, }\AttributeTok{alpha =} \FloatTok{0.05}\NormalTok{, }\AttributeTok{n\_sim =} \DecValTok{1000}\NormalTok{)}
\NormalTok{  powers\_0}\FloatTok{.4} \OtherTok{=} \FunctionTok{c}\NormalTok{(powers\_0}\FloatTok{.4}\NormalTok{, power)}
\NormalTok{\}}
\NormalTok{simulated\_power\_sample\_size\_0}\FloatTok{.4} \OtherTok{=} \FunctionTok{data.frame}\NormalTok{(sample\_sizes, powers\_0}\FloatTok{.4}\NormalTok{) }
\end{Highlighting}
\end{Shaded}

\begin{Shaded}
\begin{Highlighting}[]
\CommentTok{\# Create visualization}
\FunctionTok{library}\NormalTok{(ggplot2)}
\CommentTok{\# Plot power vs sample size}
\NormalTok{graph\_simulated\_power\_sample\_size }\OtherTok{=} \FunctionTok{ggplot}\NormalTok{() }\SpecialCharTok{+}
  \FunctionTok{geom\_line}\NormalTok{(}\AttributeTok{data =}\NormalTok{ simulated\_power\_sample\_size\_0}\FloatTok{.2}\NormalTok{, }\FunctionTok{aes}\NormalTok{(}\AttributeTok{x =}\NormalTok{ sample\_sizes, }\AttributeTok{y =}\NormalTok{ powers\_0}\FloatTok{.2}\NormalTok{), }\AttributeTok{color =} \StringTok{"blue"}\NormalTok{) }\SpecialCharTok{+}
  \FunctionTok{geom\_line}\NormalTok{(}\AttributeTok{data =}\NormalTok{ simulated\_power\_sample\_size\_0}\FloatTok{.4}\NormalTok{, }\FunctionTok{aes}\NormalTok{(}\AttributeTok{x =}\NormalTok{ sample\_sizes, }\AttributeTok{y =}\NormalTok{ powers\_0}\FloatTok{.4}\NormalTok{), }\AttributeTok{color =} \StringTok{"red"}\NormalTok{) }\SpecialCharTok{+} 
  \FunctionTok{labs}\NormalTok{(}\AttributeTok{x =} \StringTok{"Simulated Power"}\NormalTok{, }\AttributeTok{y =} \StringTok{"Sample Size"}\NormalTok{, }\AttributeTok{title =} \StringTok{"Simulated power by sample size with effect sizes 0.2  (blue) and 0.4 (red)"}\NormalTok{)}

\NormalTok{graph\_simulated\_power\_sample\_size}
\end{Highlighting}
\end{Shaded}

\pandocbounded{\includegraphics[keepaspectratio]{ps7_files/figure-pdf/problem2a4_visualizing-1.pdf}}

(I do not go for creativity in labeling this graph).

\textbf{Questions:} 1. What sample size is needed to achieve 80\% power
for detecting an effect of 0.2?

You need at least 400 observations.

\begin{enumerate}
\def\labelenumi{\arabic{enumi}.}
\setcounter{enumi}{1}
\tightlist
\item
  How does changing the true effect size to 0.4 affect the required
  sample size?
\end{enumerate}

The required sample size goes way down, from 400 to 100; doubling the
effect reduces the required sample size by a factor of four.

\begin{enumerate}
\def\labelenumi{\arabic{enumi}.}
\setcounter{enumi}{2}
\tightlist
\item
  What happens to power if you double the variance (sigma)?
\end{enumerate}

Let's find out!

\begin{Shaded}
\begin{Highlighting}[]
\NormalTok{count }\OtherTok{=} \DecValTok{0}
\NormalTok{powers\_0}\FloatTok{.2}\NormalTok{\_sigma2 }\OtherTok{=} \FunctionTok{c}\NormalTok{()}
\ControlFlowTok{for}\NormalTok{ (i }\ControlFlowTok{in}\NormalTok{ sample\_sizes)\{}
\NormalTok{  count }\OtherTok{=}\NormalTok{ count }\SpecialCharTok{+} \DecValTok{1}
\NormalTok{  power }\OtherTok{=} \FunctionTok{simulate\_power}\NormalTok{(}\AttributeTok{true\_effect =} \FloatTok{0.2}\NormalTok{, }\AttributeTok{sample\_size =}\NormalTok{ sample\_sizes[count], }\AttributeTok{sigma =} \DecValTok{2}\NormalTok{, }\AttributeTok{alpha =} \FloatTok{0.05}\NormalTok{, }\AttributeTok{n\_sim =} \DecValTok{1000}\NormalTok{)}
\NormalTok{  powers\_0}\FloatTok{.2}\NormalTok{\_sigma2 }\OtherTok{=} \FunctionTok{c}\NormalTok{(powers\_0}\FloatTok{.2}\NormalTok{\_sigma2, power)}
\NormalTok{\}}
\NormalTok{simulated\_power\_sample\_size\_0}\FloatTok{.2}\NormalTok{\_sigma2 }\OtherTok{=} \FunctionTok{data.frame}\NormalTok{(sample\_sizes, powers\_0}\FloatTok{.2}\NormalTok{\_sigma2)}

\NormalTok{count }\OtherTok{=} \DecValTok{0}
\NormalTok{powers\_0}\FloatTok{.4}\NormalTok{\_sigma2 }\OtherTok{=} \FunctionTok{c}\NormalTok{()}
\ControlFlowTok{for}\NormalTok{ (i }\ControlFlowTok{in}\NormalTok{ sample\_sizes)\{}
\NormalTok{  count }\OtherTok{=}\NormalTok{ count }\SpecialCharTok{+} \DecValTok{1}
\NormalTok{  power }\OtherTok{=} \FunctionTok{simulate\_power}\NormalTok{(}\AttributeTok{true\_effect =} \FloatTok{0.4}\NormalTok{, }\AttributeTok{sample\_size =}\NormalTok{ sample\_sizes[count], }\AttributeTok{sigma =} \DecValTok{2}\NormalTok{, }\AttributeTok{alpha =} \FloatTok{0.05}\NormalTok{, }\AttributeTok{n\_sim =} \DecValTok{1000}\NormalTok{)}
\NormalTok{  powers\_0}\FloatTok{.4}\NormalTok{\_sigma2 }\OtherTok{=} \FunctionTok{c}\NormalTok{(powers\_0}\FloatTok{.4}\NormalTok{\_sigma2, power)}
\NormalTok{\}}
\NormalTok{simulated\_power\_sample\_size\_0}\FloatTok{.4}\NormalTok{\_sigma2 }\OtherTok{=} \FunctionTok{data.frame}\NormalTok{(sample\_sizes, powers\_0}\FloatTok{.4}\NormalTok{\_sigma2) }
\end{Highlighting}
\end{Shaded}

\begin{Shaded}
\begin{Highlighting}[]
\CommentTok{\# Plot power vs sample size}
\NormalTok{colors }\OtherTok{=} \FunctionTok{c}\NormalTok{(}\StringTok{"powers\_0.2"} \OtherTok{=} \StringTok{"blue"}\NormalTok{, }\StringTok{"powers\_0.2\_sigma2"} \OtherTok{=} \StringTok{"lightblue"}\NormalTok{, }\StringTok{"powers\_0.4"} \OtherTok{=} \StringTok{"red"}\NormalTok{, }\StringTok{"powers\_0.4\_sigma2"} \OtherTok{=} \StringTok{"pink"}\NormalTok{)}

\NormalTok{graph\_simulated\_power\_sample\_size\_and\_variance }\OtherTok{=} \FunctionTok{ggplot}\NormalTok{() }\SpecialCharTok{+}
  \FunctionTok{scale\_color\_manual}\NormalTok{(}\AttributeTok{values =}\NormalTok{ colors, }\AttributeTok{labels =} \FunctionTok{c}\NormalTok{(}\StringTok{"Effect of 0.2 and sigma of 1"}\NormalTok{, }\StringTok{"Effect of 0.2 and sigma of 2"}\NormalTok{, }\StringTok{"Effect of 0.4 and sigma of 1"}\NormalTok{, }\StringTok{"Effect of 0.4 and sigma of 2"}\NormalTok{)) }\SpecialCharTok{+}
  \FunctionTok{geom\_line}\NormalTok{(}\AttributeTok{data =}\NormalTok{ simulated\_power\_sample\_size\_0}\FloatTok{.2}\NormalTok{, }\FunctionTok{aes}\NormalTok{(}\AttributeTok{x =}\NormalTok{ sample\_sizes, }\AttributeTok{y =}\NormalTok{ powers\_0}\FloatTok{.2}\NormalTok{, }\AttributeTok{color =} \StringTok{"powers\_0.2"}\NormalTok{)) }\SpecialCharTok{+}
  \FunctionTok{geom\_line}\NormalTok{(}\AttributeTok{data =}\NormalTok{ simulated\_power\_sample\_size\_0}\FloatTok{.2}\NormalTok{\_sigma2, }\FunctionTok{aes}\NormalTok{(}\AttributeTok{x =}\NormalTok{ sample\_sizes, }\AttributeTok{y =}\NormalTok{ powers\_0}\FloatTok{.2}\NormalTok{\_sigma2, }\AttributeTok{color =} \StringTok{"powers\_0.2\_sigma2"}\NormalTok{)) }\SpecialCharTok{+} 
  \FunctionTok{geom\_line}\NormalTok{(}\AttributeTok{data =}\NormalTok{ simulated\_power\_sample\_size\_0}\FloatTok{.4}\NormalTok{, }\FunctionTok{aes}\NormalTok{(}\AttributeTok{x =}\NormalTok{ sample\_sizes, }\AttributeTok{y =}\NormalTok{ powers\_0}\FloatTok{.4}\NormalTok{, }\AttributeTok{color =} \StringTok{"powers\_0.4"}\NormalTok{))}\SpecialCharTok{+} 
    \FunctionTok{geom\_line}\NormalTok{(}\AttributeTok{data =}\NormalTok{ simulated\_power\_sample\_size\_0}\FloatTok{.4}\NormalTok{\_sigma2, }\FunctionTok{aes}\NormalTok{(}\AttributeTok{x =}\NormalTok{ sample\_sizes, }\AttributeTok{y =}\NormalTok{ powers\_0}\FloatTok{.4}\NormalTok{\_sigma2, }\AttributeTok{color =} \StringTok{"powers\_0.4\_sigma2"}\NormalTok{)) }\SpecialCharTok{+} 
  \FunctionTok{labs}\NormalTok{(}
    \AttributeTok{x =} \StringTok{"Simulated Power"}\NormalTok{, }
    \AttributeTok{y =} \StringTok{"Sample Size"}\NormalTok{, }
    \AttributeTok{title =} \StringTok{"Simulated power by sample size with effect sizes 0.2 and 0.4"}
\NormalTok{    )}

\NormalTok{graph\_simulated\_power\_sample\_size\_and\_variance}
\end{Highlighting}
\end{Shaded}

\pandocbounded{\includegraphics[keepaspectratio]{ps7_files/figure-pdf/unnamed-chunk-2-1.pdf}}

So what we can see from this graph is that doubling the sigma basically
cancels out the effect of doubling the effect. A sample size four times
larger is required when variance is doubled.

\subsection{2b.}\label{b.}

\textbf{Questions:} 1. What is the ``winner's curse'' and why does it
occur?

A problem that comes with insufficient power is not just that we will
make Type II errors and fail to distinguish a real effect from zero. We
also find that when our confidence intervals are very wide (because of a
small sample size or large variance) and our predicted effect fairly
small, any ``successes'' we get even in the right direction will always
be massive overestimates. With large confidence intervals, small true
effects are indistinguishable from zero.

\begin{enumerate}
\def\labelenumi{\arabic{enumi}.}
\setcounter{enumi}{1}
\tightlist
\item
  How does sample size affect the magnitude of the winner's curse?
\end{enumerate}

If you can increase your sample size, you reduce the magnitude of the
problem, since you become better able to identify smaller real effects
as your confidence intervals shrink.

\begin{center}\rule{0.5\linewidth}{0.5pt}\end{center}

\section{Problem 3}\label{problem-3}

\subsection{3a.}\label{a.-1}

\begin{enumerate}
\def\labelenumi{\arabic{enumi}.}
\tightlist
\item
  Define and distinguish between:

  \begin{itemize}
  \tightlist
  \item
    Moderator variable
  \item
    Mediator variable
  \end{itemize}
\item
  Draw path diagrams for both (like in the slides).
\end{enumerate}

Okay I got no clue how to do this in R. A doodle is attached.
\pandocbounded{\includegraphics[keepaspectratio]{./Users/gideon/Downloads/A_moderator_variable.png}}

\begin{figure}[H]

{\centering \pandocbounded{\includegraphics[keepaspectratio]{./Users/gideon/Downloads/A_mediator_variable.png}}

}

\caption{A mediator variable}

\end{figure}%

\begin{enumerate}
\def\labelenumi{\arabic{enumi}.}
\setcounter{enumi}{2}
\tightlist
\item
  Provide a political science example of each.
\end{enumerate}

\subsection{3b.}\label{b.-1}

Using the QOG data from the slides:

\begin{Shaded}
\begin{Highlighting}[]
\FunctionTok{library}\NormalTok{(rqog)}
\FunctionTok{library}\NormalTok{(dplyr)}
\FunctionTok{library}\NormalTok{(ggplot2)}

\CommentTok{\# Load and prepare data}
\NormalTok{qog\_data }\OtherTok{\textless{}{-}} \FunctionTok{read\_qog}\NormalTok{(}\AttributeTok{which\_data =} \StringTok{"standard"}\NormalTok{, }\AttributeTok{data\_type =} \StringTok{"time{-}series"}\NormalTok{)}

\CommentTok{\# Create analysis dataset}
\NormalTok{analysis\_data }\OtherTok{\textless{}{-}}\NormalTok{ qog\_data }\SpecialCharTok{\%\textgreater{}\%}
  \FunctionTok{select}\NormalTok{(}
    \AttributeTok{country =}\NormalTok{ cname,}
    \AttributeTok{year =}\NormalTok{ year,}
    \AttributeTok{democracy =}\NormalTok{ vdem\_libdem,}
    \AttributeTok{gdp\_pc =}\NormalTok{ gle\_cgdpc,}
    \AttributeTok{colonial =}\NormalTok{ ht\_colonial}
\NormalTok{  ) }\SpecialCharTok{\%\textgreater{}\%}
  \FunctionTok{filter}\NormalTok{(}\SpecialCharTok{!}\FunctionTok{is.na}\NormalTok{(democracy), }\SpecialCharTok{!}\FunctionTok{is.na}\NormalTok{(gdp\_pc), }\SpecialCharTok{!}\FunctionTok{is.na}\NormalTok{(colonial)) }\SpecialCharTok{\%\textgreater{}\%}
  \FunctionTok{group\_by}\NormalTok{(country) }\SpecialCharTok{\%\textgreater{}\%}
  \FunctionTok{filter}\NormalTok{(year }\SpecialCharTok{==} \FunctionTok{max}\NormalTok{(year)) }\SpecialCharTok{\%\textgreater{}\%}
  \FunctionTok{ungroup}\NormalTok{() }\SpecialCharTok{\%\textgreater{}\%}
  \FunctionTok{mutate}\NormalTok{(}
    \AttributeTok{log\_gdp =} \FunctionTok{log}\NormalTok{(gdp\_pc),}
    \AttributeTok{colonized =} \FunctionTok{ifelse}\NormalTok{(colonial }\SpecialCharTok{\textgreater{}} \DecValTok{0}\NormalTok{, }\DecValTok{1}\NormalTok{, }\DecValTok{0}\NormalTok{)}
\NormalTok{  )}

\CommentTok{\# Your tasks:}
\CommentTok{\# 1. Run two models:}
\CommentTok{\#    a. Main effects only: democracy \textasciitilde{} log\_gdp + colonized}
\CommentTok{\#    b. With interaction: democracy \textasciitilde{} log\_gdp * colonized}

\CommentTok{\# 2. Calculate and interpret:}
\CommentTok{\#    a. The marginal effect of log\_gdp when colonized = 0}
\CommentTok{\#    b. The marginal effect of log\_gdp when colonized = 1}
\CommentTok{\#    c. Test whether these effects are statistically different}

\CommentTok{\# 3. Create visualization:}
\CommentTok{\#    a. Plot with two regression lines (one for each colonized status)}
\CommentTok{\#    b. Include confidence bands}
\CommentTok{\#    c. Add appropriate labels and title}
\end{Highlighting}
\end{Shaded}

\textbf{Questions:} 1. How does the relationship between GDP and
democracy differ between former colonies and never-colonized countries?

\begin{enumerate}
\def\labelenumi{\arabic{enumi}.}
\setcounter{enumi}{1}
\tightlist
\item
  Is the interaction statistically significant? What does this mean
  substantively?
\end{enumerate}




\end{document}
